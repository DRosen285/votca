\begin{itemize}
\item \textbf{cgmap} \\
      Mapping file for analysis. Currently also during coarse-grained simulations, a mappin file has to be specified
      which is only a 1:1 mapping. This will change in the future.
\item \textbf{non-bonded} \\
      Section for a non-bonded interaction. All sub-tags will be later on specified in 
      interaction.* since most of them are identical to bonded interactions.
\item \textbf{bonded} \\
      Section for a bonded interaction. All sub-tags will be later on specified in 
      interaction.* since most of them are identical to non-bonded interactions.
\item \textbf{interaction.name} \\
      Name of the interaction
\item \textbf{interaction.target} \\
 target distribution (rdf), just give gromas rdf.xvg 
\item \textbf{interaction.type1} \\
Only for non-bonded. Bead type 1 of non-bonded interaction.
\item \textbf{interaction.type2} \\
Only for non-bonded. Bead type 2 of non-bonded interaction.
\item \textbf{interaction.min} \\
lower bound of interval for potential table in which calculations are performed.
\item \textbf{interaction.max} \\
upper bound of interval for potential table in which calculations are performed.
\item \textbf{interaction.step} \\
step size of interval for potential table in which calculations are performed.
\item \textbf{interaction.do\_potential} \\
Update cacly for potential update. 1 means update, 0 don't update. 1 1 0 means update 2 iterations, then don't update, then repeated. 
\item \textbf{interaction.post\_update} \\
Additional post processing of dU before added to potential. This is a list of scripts which are called separated by space.
\item \textbf{interaction.post\_update\_options} \\
This sectioncontains all options for post update scripts.
\item \textbf{interaction.post\_add} \\
Additional post processing of U after dU added to potential. See interaction.post\_update for details.
\item \textbf{interaction.imc} \\
Section containing inverse monte carlo specific options.
\item \textbf{interaction.imc.group} \\
      Group of interaction. Cross-correlations of all members of a group are taken into account for calculating the update.
        I no cross correlations should be calculated, interactions have to be put into different groujps.
\item \textbf{interaction.gromacs} \\
        This section contains gromacs specific options if gromacs is used as simulatiion program.
\item \textbf{interaction.gromacs.table} \\
      Name of file for tabulated potential of this interaction. This fill will be created from the internal tabulated potential format for every run.
\item \textbf{inverse} \\
 general options for inverse script 
\item \textbf{inverse.p\_target} \\
 target pressure 
\item \textbf{inverse.kBT} \\
 kBT (300*0.00831451 gromacs units) 
\item \textbf{inverse.program} \\
simulation package to be used (currently only gromacs) 
\item \textbf{inverse.gromacs} \\
 gromacs specific options 
\item \textbf{inverse.gromacs.equi\_time} \\
 trash so many frames at the beginning 
\item \textbf{inverse.gromacs.table\_bins} \\
 grid for table*.xvg !
\item \textbf{inverse.gromacs.pot\_max} \\
 cut the potential at this value (gromacs bug) 
\item \textbf{inverse.gromacs.table\_extend} \\
 extend the tables to this value 
\item \textbf{inverse.gromacs.topol} \\
 topology + trajectory. Be careful, do not change yet! 
\item \textbf{inverse.gromacs.traj} \\
 topology + trajectory. Be careful, do not change yet! 
\item \textbf{inverse.filelist} \\
 these files are copied for each new run 
\item \textbf{inverse.iterations\_max} \\
 do so many iterations 
\item \textbf{inverse.method} \\
 ibm: inverse biltzmann imc: inverse monte carlo 
\item \textbf{inverse.scriptdir} \\
 directory for user scripts (e.g. \$PWD)
\item \textbf{inverse.log\_file} \\
 write log to this file 
\item \textbf{inverse.imc} \\
 general imc specific options 
\item \textbf{inverse.imc.solver} \\
 solver to solve linear equation system, can be octave or matlab 
\end{itemize}
