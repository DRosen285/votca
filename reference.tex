\chapter{Reference}
\section{Programs}
\label{sec:ref_programs}
\input{reference/programs/all}
\section{Mapping file}
\label{sec:ref_mapping}
The root node always has to be cg\_molecule. It can contain the following keywords:

\section{Mapping definitions}
{\em Mapping definitions} describe how to map a single molecule from atomistic to coarse-grained representation. The mapping definition have only to be specified once per molecule. The file contains sections for coarse-grained beads, bonded interactions in coarse grained scheme as well as mapping matrices. 

\subsection{Structure of mapping file}
The root node always has to be cg\_molecule. It can contain the following keywords:


\begin{itemize}
\item \textbf{ident} \\
      Molecule name in reference topology.
\item \textbf{maps} \\
      Section containing definitions of mapping schemes.
\item \textbf{maps.map} \\
      Section for a mapping for 1 bead.
\item \textbf{maps.map.name} \\
      Name of the mapping.
\item \textbf{maps.map.weights} \\
      Weights of the mapping matrix. Entries are normalized to 1, number of entries must match the number of reference beads in a coarse-grained bead.
\item \textbf{name} \\
      Name of molecule in coarse-grained representation.
\item \textbf{topology} \\
      Section containing definition of coarse grained topology of molecule.
\item \textbf{topology.cg\_beads} \\
      Section defining coarse grained beads of molecule.
\item \textbf{topology.cg\_beads.cg\_bead} \\
      Definition of a coarse grained bead.
\item \textbf{topology.cg\_beads.cg\_bead.beads} \\
      The beads section lists all atoms of the reference system that are mapped to this particular
      coarse grained bead. The syntax is RESID:RESNAME:ATOMNAME, the beads are seperated by spaces.
\item \textbf{topology.cg\_beads.cg\_bead.mapping} \\
      Mapping scheme to be used for this bead (specified in section mapping) to map from reference system.
\item \textbf{topology.cg\_beads.cg\_bead.name} \\
      Name of coarse grained bead.
\item \textbf{topology.cg\_beads.cg\_bead.type} \\
      Type of coarse grained bead.
\item \textbf{topology.cg\_bonded} \\
      The cg\_bonded section contains all bonded interaction of the molecule. That can be bond, angle or dihedral.
      An entry for each group of bonded interaction can be specified, e.g. several groups (types) of bonds can be specified.
      A specific bonded interaction can be later on addressed by MOLECULE:NAME:NUMBER, where MOLECULE is the molecule ID in
      the whole topology, NAME the name of the interaction group and NUMBER addresses the interaction in the group.
\item \textbf{topology.cg\_bonded.angle} \\
      Definition of a group of angles.
\item \textbf{topology.cg\_bonded.angle.beads} \\
      List of triples of beads that define a bond. Names specified in cg\_beads, separated by commas.
\item \textbf{topology.cg\_bonded.angle.name} \\
      Name of the group.
\item \textbf{topology.cg\_bonded.bond} \\
      Definition of a group of bonds.
\item \textbf{topology.cg\_bonded.bond.beads} \\
      List of pair of beads that define a bond. Names specified in cg\_beads, separated by commas.
\item \textbf{topology.cg\_bonded.bond.name} \\
      Name of the group.
\item \textbf{topology.cg\_bonded.dihedral} \\
      Definition of a group of dihedrals. Since the exact functional form does not matter, this combines proper as well as improper dihedrals.
\item \textbf{topology.cg\_bonded.dihedral.beads} \\
      List of quadruples of beads that define a bond. Names specified in cg\_beads, separated by commas.
\item \textbf{topology.cg\_bonded.dihedral.name} \\
      Name of the group.
\end{itemize}


\subsection{Example - mapping file for propane}

\lstinputlisting[frame=single,caption=Mapping for propane]{usage/propane.xml}


\section{Settings file}
All options for the iterative script are stored in an xml file.
\label{sec:ref_options}
\input{reference/xml/cgoptions.xml}

\subsection{Interaction options}
\label{sec:ref_interaction}
This section contains all interaction option, which could be contained in the \cgopt{non-bonded} or \cgopt{bonded} section in \sect{sec:ref_options}.
\input{reference/xml/cginteraction.xml}
\vfill

\section{Scripts}
\label{sec:csg_table}
Scripts are used by \prog{csg_call} and \prog{csg_inverse}.
The script table commonly used (compare \texttt{csg\_call --list}): 
\input{reference/scripts/csg_table}
Script calls can be overwritten by adding a line with the 3rd column changed to \texttt{csg\_table} in \cgopt{inverse.scriptdir} directory.
\input{reference/scripts/all}
