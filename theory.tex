\chapter{Theoretical background}

\section{Mapping operator}
\label{sec:mapping_operator}
%\sasha
%mapping scheme ($c_{Ii}$ coefficients) \\
TO DO: picture \\
%translation invariance \\
%definition of the mass \\
%definition of specific and involved atoms \\

The mapping operator establishes a link between the atomistic and coarse-grained representations of the system. An atomistic system is described by specifying the values of the Cartesian coordinates and momenta  
\begin{eqnarray}
\bm r^n &=& \{\bm r_1,\dots,\bm r_n\}, \\
\bm p^n &=& \{\bm p_1,\dots,\bm p_n\}.
\end{eqnarray}
of the $n$ atoms in the system.\footnote{In what follows we adopt notations of ref.~\cite{Noid:2008.1}.} 
%
On a coarse-grained level, the coordinates and momenta are specified by the positions and momenta of CG sites 
\begin{eqnarray}
\bm R^N = \{\bm R_1,\dots,\bm R_N\}, \\ 
\bm P^N = \{\bm P_1,\dots,\bm P_N\}.
\end{eqnarray}
Note that capitalized symbols are used for the CG sites while lower case letters are used for the atomistic system.

The mapping operator defined by a matrix for each bead $I$, ${\bm c}_I$, then links the two descriptions
\begin{eqnarray}
 {\bm R}_I &=& \sum_{i=1}^{n}c_{Ii}\bm r_i, \\
 {\bm P}_I &=& 
 	M_I \dot{{\bm R}}_I = 
	M_I \sum_{i=1}^{n}c_{Ii} \dot{{\bm r}}_i = 
	M_I \sum_{i=1}^{n} \frac{ c_{Ii}} {m_i} {\bm p}_i . 
\label{eq:mapping_scheme}
\end{eqnarray}
for all $I = 1,\dots,N$.

If an atomistic system is translated by a constant vector, the corresponding coarse-grained system is also translated by the same vector. This implies that, for all $I$,
\begin{equation}
 \sum_{i=1}^{n}c_{Ii}=1.
\end{equation}

In some cases it is useful to define CG mapping in a way that certain atoms belong to several CG beads at the same time~\cite{Fritz:2009}. Following ref.~\cite{Noid:2008.1} we define two sets of atoms for each of the $N$ CG beads. For each site $I$, a set of {\em involved} atoms is defined as
\begin{equation}
 {\cal I}_I=\{i|c_{Ii}\ne0\}.
\end{equation}
An atom $i$ in the atomistic model is involved in a CG site, \textit{I}, if and only if this atom provides a nonzero contribution to the sum in eq.~\ref{eq:mapping_scheme}.

A set of {\em specific} atoms is defined as
\begin{equation}
 {\cal S}_I=\{i|c_{Ii}\ne0 \text{ and } c_{Ji}=0 \text{ for all } J \ne I\}.
\end{equation}
In other words, atom $i$ is specific to site $I$ if and only if this atom is involved in site $I$ and is not involved in the definition of any other site.

The CG model will generate an equilibrium distribution of momenta that is consistent with an underlying atomistic model if all the atoms are {\em specific} and if the mass of the $I^\text{th}$ CG site is given by~\cite{Noid:2008.1}
\begin{equation}
M_I= \left( \sum_{i \in {\cal I}_I}\frac{c_{Ii}^2}{m_i} \right)^{-1}.
\end{equation}

%%%%%%%%%%%%%%%%%%%%%%%%%%%%%%%%%%%%%%%%%%%%%%%%%%%%%%%%%%%%%%%%%%%%%%%%%%%%%%%%%%%%%%%%%%%%%%%%
\section{Boltzmann inversion}
\label{sec:bi}

Boltzmann inversion is the simplest method one can use to obtain coarse-grained potentials~\cite{Tschoep:1998}. It is mostly used for {\em bonded} potentials, such as bonds, angles, and torsions. Boltzmann inversion is structure-based and only requires positions of atoms.

The coarse-grained potential is calculated by inverting the probability distribution $P(q)$ of a variable $q$, which is either a bond length, bond angle, or torsion angle
\begin{equation}
  U(q) = - k_\text{B} T \ln  P(q) ~.
  \label{eq:inv_boltzmann}
\end{equation}
%
Note that the normalization factor $Z$ is not important since it would only enter the coarse-grained potential $U(q)$ as an irrelevant additive constant.

Histograms for the bonds $H_r(r)$, angles $H_\theta(\theta)$, and torsion angles $H_\varphi(\varphi)$ have to be rescaled to obtain the volume normalized distribution functions $P_r(r)$, $P_\theta(\theta)$, and $P_\varphi(\varphi)$, respectively: 
%
\begin{align}
    P_r(r) = \frac{H_r(r)}{4\pi r^2}~,
    P_\theta(\theta) = \frac{H_\theta(\theta)}{\sin \theta}~,
    P_\varphi(\varphi) = H_\varphi (\varphi)~.
    \label{eq:boltzmann_norm}
\end{align}
with bond length $r$, angle~$\theta$ and torsion angle~$\varphi$.%
The coarse-grained potential can then be calculated by Boltzmann inversion of the distribution functions
%
\begin{align}
    \label{eq:boltzmann_pmf}
    U({r}, \theta, \varphi) &= U_r({r}) + U_{\theta}(\theta) + U_{\varphi}(\varphi)~, \\
    U_q({q}) &= - k_\text{B} T \ln P_q( q ),\; q=r, \theta, \varphi~.
    \nonumber
\end{align}


%%%%%%%%%%%%%%%%%%%%%%%%%%%%%%%%%%%%%%%%%%%%%%%%%%%%%%%%%%%%%%%%%%%%%%%%%%%%%%%%%%%%%%%%%%%%%%%%
\section{Iterative Boltzmann Inversion}
\label{sec:ibi}

In \ibi~\cite{Reith:2003}, potentials are refined iteratively. The potential update $\Delta U$ is given by
\begin{eqnarray}
  \label{eq:iter_boltzmann}
  U^{(n+1)} &=& U^{(n)} + \lambda \Delta U^{(n)}~, \\
  \Delta U^{(n)} &=&  k_\text{B} T \ln  \frac{P^{(n)}}{P_{\rm ref}}
  =  U_\text{PMF}^\text{ref} - U_\text{PMF}^{(n)}~.
\end{eqnarray}
Here $\lambda \in ]0,1]$ is a numerical factor to stabilize the scheme.

\section{Inverse Monte Carlo}
\label{sec:imc}

\imc is another iterative scheme, which additionally includes cross correlations. The potential update $\Delta U$ is calculated by solving a set of linear equations
\begin{align}
    \left<S_{\alpha}\right> - S_{\alpha}^{\text{ref}}= A_{\alpha \gamma} \Delta U_{\gamma}~,
  \label{eq:imc}
\end{align}
%
with
\begin{eqnarray}
  \label{eq:covariance}
  A_{\alpha \gamma} &=& \frac{\partial \left< S_{\alpha} \right> }{\partial U_{\gamma}}  \\
  \nonumber
  &=&
  \beta \left( \left<S_{\alpha} \right>\left<S_{\gamma} \right> - \left<S_{\alpha} S_{\gamma} \right>  \right)~.
  \nonumber
\end{eqnarray}
and $S$ the histogram of the interaction. Here $S$ and $H$ from Boltzmann Inversion both mean the same, due to notation used in the original papers we keep these separat labels.

\section{Force Matching}
\label{sec:fm}

\sasha

Brief description with references \\
Maybe appendix with main equations \\

Force matching (\fm) is another approach to evaluate corse-grained potentials~\cite{Ercolessi:1994,Izvekov:2005,Noid:2007}. In contrast to the structure-based approaches, its aim is not to reproduce various distribution functions, but instead to match the multibody potential of mean force as close as possible with a given set of coarse grained interactions. 

The method works as follows. We first assume that the coarse-grained force-field (and hence the forces) depends on $M$ parameters $g_1,...,g_M $. These parameters can be prefactors of analytical functions, tabulated values of the interaction potentials, or coefficients of splines used to describe these potentials.

In order to determine these parameters, the reference forces on coarse-grained beads are calculated by simply summing up the forces on the atoms
\begin{equation}
  {\vec F}_I^\text{ref} = \sum_{j \in {\cal S_I}} {\vec f}_j({\vec r^n}),
  \label{eq:force_mapping}
\end{equation}
where the sum goes over all atoms {\it specific} to the CG site {\it I} (see. \sect{sec:mapping_operator}).
This is the default behavior. One can also generalize this by including $d_{Ij}$ coefficients in the force mapping \cite{Noid:2008.1}. In order to do that one should add a d-coefficients tag to the mapping file, see. \sect{sec:mapping}

By calculating the reference forces for $L$ snapshots we can write down $N \times L$ equations
%
\begin{equation}
  {\vec F}_{il}^\text{cg}(g_1,...,g_M)=\vec F_{il}^\text{ref},\, i=1,\dots,N, l=1,\dots,L~.
  \label{eq:fmatch1}
\end{equation}
%
Here ${\vec F}_{il}^\text{ref}$ is the force on the bead $i$, ${\vec F}_{il}^\text{cg} $ is the coarse-grained representation of this force. Index $l$ enumerates snapshots picked for coarse-graining. By running the simulations long enough one can always ensure that $M < N \times L$. In this case the set of equations~\ref{eq:fmatch1} is overdetermined and can be solved in a least-squares manner.

Going back to the set of equations~\ref{eq:fmatch1} one can see that ${\bm F}_{il}^\text{cg}$ is, in principle, a non-linear function of its parameters $\{g_i\}$. It is, therefore, useful to represent the coarse-grained force-field in such a way that equations~(\ref{eq:fmatch1}) become linear functions of $\{g_i\}$. This can be done using splines to describe the functional form of the forces~\cite{Izvekov:2005}.

An adequate sampling of the system requires a large number of snapshots $L$. Hence, the applicability of the method is often constrained by the amount of memory available. To remedy the situation, one can split the trajectory into blocks, find the coarse-grained potential for each block and then perform averaging over the blocks. 

More details about spline implementation is available in the Appendix of the \votca paper.
