\chapter{Inverse Monte Carlo}
In this section, additional options are described to run \imc coarse graining. The usage of \imc is similar to the one of \ibi and understanding the use of the scripting framework described in chapter~\ref{sec:iterative_workflow} is necessary.

\textbf{WARNING: multicomponent \imc is still experimental!}

\section{General considerations}
In comparison to \ibi, \imc needs a significantly more statistics to calculate the potential update\cite{Ruehle:2009.a}. It is advisable to perform smoothing on the potential update. Smoothing can be performed as described in \sect{ref:ibi:optimize}. In addition, \imc can lead to problems related to finite size: for methanol, an undersized system proved to lead to a linear shift in the potential\cite{Ruehle:2009.a}. It is therefore always necessary to check that the system size is sufficiently large and that runlengh csg smoothing iterations are well balanced.

\section{Additional mapping for statistics}
The program \prog{csg_stat} is used for evaluating the \imc matrix. Although the matrix only acts on the coarse-grained system here, it still needs a mapping file to work. This will improve with one of the next releases to simplify the setup. The mapping file needs to be a one to one mapping of the coarse grained system, e.g. for coarse graining \spce water, the mapping file looks as follows:
\begin{lstlisting}
  </cg_molecule>
    <name>SOL</name> 
    <ident>SOL</ident>
    <topology>
      <cg_beads>
        <cg_bead>
          <name>CG</name>
          <type>CG</type>
          <mapping>A</mapping>
          <beads>
            1:SOL:CG 
          </beads>
        </cg_bead>
      </cg_beads>
    </topology>
    <maps>
      <map>
        <name>A</name>
        <weights>1</weights>
      </map>
    </maps>
  </cg_molecule>
\end{lstlisting}



\section{Correlation groups}
Unlike \ibi, \imc also takes cross-correlations of interactions into account in order to calculate the update. However, it might not always be beneficial to evaluate cross-correlations of all pairs of interactions. By specifying \interopt{inverse.imc.group}, \votca allows to define groups of interactions, amongst which cross-correlations are taken into account, where \interopt{inverse.imc.group} can be any name.

\begin{lstlisting}
  <non-bonded>
    <name>CG-CG</name>
    <type1>CG</type1>
    <type2>CG</type2>
    ...
    <imc>
      <group>solvent</group>
   </imc>
  </non-bonded>
  <non-bonded>
\end{lstlisting}
