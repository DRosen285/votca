\chapter{Implementation}
\subsection{Coarse-graining engine}
\votca can be divided into two main parts. A C++ kernel, which includes various coarse-graining tools and is capable of processing topologies and trajectories, and a scripting enginge to steers the iterative procedures.

These tools include calculations of probability distributions of bonded and non-bonded interactions, correlation and autocorrelation functions, and updates for the coarse-grained pair potential. Analysis tools of the MD package can also be integrated into the coarse-graining workflow, as needed.

The package offers a flexible framework for reading, manipulating and analyzing of MD/SD/MC topologies and trajectories. Its core is modular and new file formats can be integrated without changing the existing code. Currently, an interface for \gromacs~\cite{gromacs4} topologies and trajectories is provided.

The coarse-graining procedure itself is controlled by several Extensible Markup Language (\xml) input files, which contain mapping and other options required for the workflow control. For the mapping, it is possible to select groups of interactions which are used for the coarse-graininged analysis. 
%It is possible to generate exclusion lists for Boltzmann inversion of bonded interactions or virtual sites for back-mapping.

\section{Mapping definitions}
{\em Mapping definitions} describe how to map a single molecule from atomistic to coarse-grained representation. The mapping definition have only to be specified once per molecule. The file contains sections for coarse-grained beads, bonded interactions in coarse grained scheme as well as mapping matrices. 

\subsection{Structure of mapping file}
The root node always has to be cg\_molecule. It can contain the following keywords:


\begin{itemize}
\item \textbf{ident} \\
      Molecule name in reference topology.
\item \textbf{maps} \\
      Section containing definitions of mapping schemes.
\item \textbf{maps.map} \\
      Section for a mapping for 1 bead.
\item \textbf{maps.map.name} \\
      Name of the mapping.
\item \textbf{maps.map.weights} \\
      Weights of the mapping matrix. Entries are normalized to 1, number of entries must match the number of reference beads in a coarse-grained bead.
\item \textbf{name} \\
      Name of molecule in coarse-grained representation.
\item \textbf{topology} \\
      Section containing definition of coarse grained topology of molecule.
\item \textbf{topology.cg\_beads} \\
      Section defining coarse grained beads of molecule.
\item \textbf{topology.cg\_beads.cg\_bead} \\
      Definition of a coarse grained bead.
\item \textbf{topology.cg\_beads.cg\_bead.beads} \\
      The beads section lists all atoms of the reference system that are mapped to this particular
      coarse grained bead. The syntax is RESID:RESNAME:ATOMNAME, the beads are seperated by spaces.
\item \textbf{topology.cg\_beads.cg\_bead.mapping} \\
      Mapping scheme to be used for this bead (specified in section mapping) to map from reference system.
\item \textbf{topology.cg\_beads.cg\_bead.name} \\
      Name of coarse grained bead.
\item \textbf{topology.cg\_beads.cg\_bead.type} \\
      Type of coarse grained bead.
\item \textbf{topology.cg\_bonded} \\
      The cg\_bonded section contains all bonded interaction of the molecule. That can be bond, angle or dihedral.
      An entry for each group of bonded interaction can be specified, e.g. several groups (types) of bonds can be specified.
      A specific bonded interaction can be later on addressed by MOLECULE:NAME:NUMBER, where MOLECULE is the molecule ID in
      the whole topology, NAME the name of the interaction group and NUMBER addresses the interaction in the group.
\item \textbf{topology.cg\_bonded.angle} \\
      Definition of a group of angles.
\item \textbf{topology.cg\_bonded.angle.beads} \\
      List of triples of beads that define a bond. Names specified in cg\_beads, separated by commas.
\item \textbf{topology.cg\_bonded.angle.name} \\
      Name of the group.
\item \textbf{topology.cg\_bonded.bond} \\
      Definition of a group of bonds.
\item \textbf{topology.cg\_bonded.bond.beads} \\
      List of pair of beads that define a bond. Names specified in cg\_beads, separated by commas.
\item \textbf{topology.cg\_bonded.bond.name} \\
      Name of the group.
\item \textbf{topology.cg\_bonded.dihedral} \\
      Definition of a group of dihedrals. Since the exact functional form does not matter, this combines proper as well as improper dihedrals.
\item \textbf{topology.cg\_bonded.dihedral.beads} \\
      List of quadruples of beads that define a bond. Names specified in cg\_beads, separated by commas.
\item \textbf{topology.cg\_bonded.dihedral.name} \\
      Name of the group.
\end{itemize}


\subsection{Example - mapping file for propane}

\lstinputlisting[frame=single,caption=Mapping for propane]{usage/propane.xml}


\section{Topologies}
A topology is defined as a set of beads with a certain bead type which are connected by bonded interactions.
\votca can read topologies in the \gromacs tpr format. However, it can also use a pdb file as topology, but, not all information are given or defined in this file format. If a pdb topology is used, it is recommended to fill in any additional molecule information using the xml advanced topology handling feature (see. \sect{sec:adv_topology})
\subsection{Manual topology handling}
\label{sec:adv_topology}
Many file formats do not have a clear definition of molecules. This can lead to problems, especially if a simulation contains several molecules with multiple types. Since during the coarse graining, the molecule type is identified by a name tag, names must also be set properly. \votca also offers the possibility to read a topology and later on modify parts of it. Everything is based on a definiton in an \xml topology file. This new \xml topology can then be used as topology in the \progopt{--tpr} option, as any other topology file. 

As \gromacs did not store molecule names in previous versions, but has introduced this feature in a newer release,  a recent .tpr file which contains molecule names should always be provided. For old topologies, rerun \gromacs \textbf{grompp} if possible to get a recent file. 

If no molecule information is present at all, it has to be manually defined. An example for a topology file that reads in last.pdb and then creates proper molecule definitions is:
\begin{lstlisting}
<topology base="last.pdb">
  <molecules>
    <clear/>
    <define name="mCP" first="1" nbeads="52" nmols="216"/>
  </molecules>
</topology>
\end{lstlisting}
where $<$clear/$>$ clears all molecule information that were present before.

If molecule information is already present in the parent topology, but molecule do not have proper names, molecules can be renamed. A short example is
\begin{lstlisting}
 <topology base="topol.tpr">
   <molecules>
     <rename name="PPY3" range="1:125"/>
     <rename name="Cl" range="126:250"/>
   </molecules>
 </topology>
\end{lstlisting}
Here, first the file topol.tpr is loaded, subsequently, molecules are renamed.

\subsection{Trajectories}
A trajectory is defined as a set of frames containing coordinates and eventually velecities and forces for the beads of a topology.
\votca currently supports trr, xtc, pdb and gro as trajector formats.

\section{Iterative workflow control}
\label{sec:impl:scripting}
For each iterative method (e.g. IBI or IMC), a simulation has to be performed with a working potential. Based on statistics from that coarse grained run, an update can be calculated. \votca implements a flexible scripting framework to perform such an iterative workflow. The workflow chart is shown in \fig{fig:flowchart}. The workflow is implemented as a shell script which can, in principle, be run on all available operating systems and provides the flexibility needed to call external (or overload existing) scripts and programs written in other programming languages. An interface to read values from the steering \xml files in C++, Perl and shell is also provided.

\begin{figure}
  \includegraphics[width=0.7\columnwidth]{functionality/abb/flowchart.eps}
  \caption{
    \label{fig:flowchart}
    Block-scheme of the workflow control for the iterative methods. The most time-consuming parts are marked in red.
  }
\end{figure}

During the global initialization the initial guess for the coarse-grained potential is calculated from the reference radial distribution function or converted from a given potential guess to the internal format. The actual iterative step starts with an iteration initialization. It searches for possible checkpoints and copies and converts files from the previous step and the base directory. Then the simulation run is prepared by converting potentials to the format required by the external sampling program and actual sampling is performed. Currently, an interface with \gromacs~\cite{gromacs4} is implemented and extension to other packages is straightforward. After sampling the phasespace, potential update $\Delta U$ is calculated. Often the update requires postprocessing, such as smoothing, interpolation, extrapolation or fitting to an analytical form. A simple pressure correction~\cite{Reith:2003} can also be seen as a postprocessing of $\Delta U$, due to the fact that it only adds a linear interparticle separation function.
%
Finally, the new potential is determined and postprocessed. If the iterative process continues, the next iterative step starts to initialize.

\subsection{General usage}
To setup the environment, run:
csh, tcsh:
\begin{verbatim}
  source <csg-installation>/bin/CSGRC.csh
\end{verbatim}
\begin{verbatim}
  bash: source <csg-installation>/bin/CSGRC.bash
\end{verbatim}

Tutorials can be found in \textit{$<$csg-installation$>$/share/csg/tutor}. Just copy one of the tutorial and try
\begin{verbatim}
 csg_inverse <options.xml>
\end{verbatim}

\subsection{Run preparation}
To run the iterations, the input for the sampling program (e.g. \gromacs ) has to be prepared in order to carry out one iteration. All files for running a coarse-grained simulation except for the tabulated potentials that should be created must be present and specified in the options xml-file.

As target distribution, any table file can be given (e.g. gromacs output from g\_rdf). The program automaticaly takes care to resample the table to the correct grid spacing as specified in the .xml file.
The initial guess is created by inverting the rdf. It's located in step\_00/$<$name$>$.pot.new. To specify the initial guess for a specific interaction by hand, write the potential table to a file called $<$name$>$.pot.in in the folder where you run the iterative procedure.

To define which interactions are iteratively refined, a section for each has to be specified in the config file. A full list of parameters can be found in ref.~\ref{sec:ref_options}.

\subsection{Running the iterative process}
After all input is set up, the run can be started with
\begin{verbatim}
  csg_inverse <options.xml>
\end{verbatim}

For each iteration, a separate folder (\textit{step\_$<$iteration$>$}) is created, where \textit{step\_00} has the special meaning of the initial setup before the first iterations. For each new iterations, the files required to run the CG simulation ( as specified in the config file) are copied to the current working folder. Also the updated potentials from the last steps ( stetp\_$<$n-1$>$/$<$interaction$>$.pot.new ) are copied and used as the new working potentials ( stetp\_$<$n$>$\/$<$interaction$>$.pot.cur ).

After the preparation, the simulation run is started. First, all potentials are converted to the format of the sampling program, and the simulation is started. As soon as the run is finished, analysis programs are started and new distributions calculated ( $<$interaction$>$.dist.new ) and the update is calculated ( $<$interaction$>$.dpot.new ). Before \votca adds the update to the old potential, the update can be processed in the post\_update step. For each script that is specified in post\_update, \votca renames $<$interaction$>$.dpot.new to $<$interaction$>$.dpot.old, stores it in $<$interaction$>$.dpot.$<$a-number$>$, and calls the processing script. Each processing script us supposed tok takes the current update ( $<$interaction$>$.dpot.cur ) and write the processed update to ( $<$interaction$>$.dpot.new ). As described, \votca takes care for all the renaming. Pressure correction is implemented as a post\_update script within this framework.

After all post\_update scripts are called, the update is added to the potential and the new potential ( $<$interaction$>$.pot.new ) is written. After that, processing of the potential is performed in the post\_add step, analogous to the tasks performed in post\_update but now for the potential instead of the update.

As a summary, the standard output files of \votca for each step are listed in the following table:

\begin{tabular}{ll}
*.dist.new & distribution functions of the current step \\
*.dpot.new & the final potential update, created in calc\_update \\
*.dpot.$<$number$>$ & for each post\_update script, the current .dpot.new is saved and a new one is created\\
*.pot.cur & the current potential, the actual run was performed with \\
*.pot.new & the new potential after add step \\
*.pot.$<$number$>$ & same as dpot.$<$number$>$ just for post\_add
\end{tabular}

If some sub step within one iteration failed, additional information can be found in the log file. The name of the log file is specified in the options xml-file.

\subsection{Continue a run}
There are two scenarios when one wants to continue a run: A finished run should be extended or the run was interrupted and one needs to restart at the current point. When csg\_inverse is started, it automatically checks for a file called \textit{done} ine the current folder. If that exists, it assumes the run as done. To extend the run, increase max\_iterations in the settings file and remove the file called done. After that, csg\_inverse can be just restarted, it will recognize available steps and continues after the last one.

If a run was interrupted, csg\_inverse might not be able to restart on it's own. In that case, the easiest solution is to delete the last step and just start again. This will just redo the last step and continue. However this method is not always practical since sampling + analysis might take very long and the run just crashed due to some post processing option. To avoid redoing the whole run, VOTCA creates a file with restart points, that marks actions in the step which were already completed (like simulation, analysis, ...). The file is specified in the option \textit{restart\_????}. If specific actions should be redone, lines in that files can also removed by hand. In addition, a file \textit{done} is created int each folder for a step that already has finished.

\subsection{Customization}
\textbf{Customization of the iterative process it not well documented yet. Documentation will improve in the near future, feel free to ask on the votca mailing list for details! }

In principle, each sub step of an iteration can be changed. All scripts that are called can be redirected ( overloaded ) to user scripts. Please never ever change the \votca installation scripts. Better copy the script and redirect it's call.

The file $<$csg-install$>$/share/csg/scripts/inverse/csg\_table assigns to two given keywords a script which is called. You can change this by creating a file
csg\_table in your own directory ( specified in the Coarse graining directives ). Just add a line for each script you want to add or change. The source\_wrapper takes care that your script is called instead of the one from the standard installations. For scripting details have a look at the source, there is no developers manual so far.




